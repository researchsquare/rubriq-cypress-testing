

\title{{ Privacy, Information Acquisition, and Market Competition}\footnote{I would like to thank Jay Pil Choi, Jon Eguia, Thomas Jeitschko, Kyoo il Kim, Aleksandr Yankelevich for valuable discussions and comments}}

\author{Soo Jin Kim\footnote{Soo Jin Kim, Michigan State University, kimsoo25@msu.edu} \\ Michigan State University
}
\date{April 2017\\ \bigskip \textbf{Preliminary. Please do not circulate.}}


\documentclass[12pt]{article}
\usepackage{mathtools}
\usepackage{amssymb}
\usepackage{bbm}
\usepackage{amsmath}
\usepackage{verbatim}
\usepackage{latexsym}
\usepackage{amsmath,amsfonts,amsthm, amssymb}
\usepackage{sectsty}
\usepackage{graphicx}
\usepackage{float}
\usepackage{cases}
\usepackage{multirow}
\usepackage{arrayjob}
\usepackage[margin=0.9in]{geometry}
\usepackage{setspace}
\usepackage{lscape}
\graphicspath{ {Graph/} }
\DeclareGraphicsExtensions{.png,.jpg,.pdf}
\DeclareGraphicsRule{.jpg}{eps}{.bb}{}
%\setlength{\textheight}{20.5cm}
\setlength{\oddsidemargin}{-0.3mm}
\setlength{\textwidth}{6.7in} 
\setstretch{1.5}  
%\renewcommand{\baselinestretch}{1.5}

\usepackage{graphicx}
\usepackage{harvard}
\usepackage{color}
\usepackage[toc,page]{appendix}
\usepackage{relsize}
\usepackage{eurosym}

\renewcommand{\appendixpagename}{\Large{Appendix}}
\renewcommand{\appendixtocname}{Appendix}

%\addtolength{\hoffset}{-.6cm}
%\addtolength{\voffset}{-1.0cm}
%\addtolength{\textwidth}{1.4cm}
%\addtolength{\textheight}{1.7cm}

\newtheorem{theorem}{Theorem}
\newtheorem{acknowledgment}{Acknowledgment}
\newtheorem{algorithm}{Algorithm}
\newtheorem{axiom}{Axiom}
\newtheorem{case}{Case}
\newtheorem{claim}{Claim}
\newtheorem{conclusion}{Conclusion}
\newtheorem{condition}{Condition}
\newtheorem{conjecture}{Conjecture}
\newtheorem{corollary}{Corollary}
\newtheorem{criterion}{Criterion}
\newtheorem{definition}{Definition}
\newtheorem{example}{Example}
\newtheorem{exercise}{Exercise}
\newtheorem{lemma}{Lemma}
\newtheorem{notation}{Notation}
\newtheorem{problem}{Problem}
\newtheorem{proposition}{Proposition}
\newtheorem{remark}{Remark}
\newtheorem{solution}{Solution}
\newtheorem{summary}{Summary}


\begin{document}
	\maketitle
	\begin{center}
	 \textbf{Abstract}
	\end{center} \begin{singlespace}
	\footnotesize{This paper analyzes how personal information whose availability depends on consumer's privacy concern affects seller's market in which targeted advertising is prevalent. If a seller obtains personal information about potential customers, he is able to attract them by making more-relevant targeted ads. In that sense, how much consumer information available affects seller market competition. I focus on how a market entrant who has worse targeting technology than the incumbent is disproportionately affected by a lack of information. The main result shows that the entrant always wants to buy data to overcome his initial disadvantage. The incumbent only wants to buy data when the less amount of personal data becomes available due to more privacy concern. In equilibrium, the portion of consumers who discloses information to enjoy better-targeted ads is sufficiently large enough for the platform to sell data exclusively to the entrant at a higher price. Due to the high data price, the entrant might not earn more profit from data monopolization although he earns greater market share and revenue. The study extends the model to the case of data-driven vertical integration between the platform and seller. The platform and incumbent are integrated if consumers become more privacy-sensitive and always foreclose the unaffiliated entrant from obtaining data access. The unaffiliated entrant suffers from loss of profit from data foreclosure. If a consumer faces higher privacy nuisance cost from the entrant due to lack of reputation, the entrant is more likely to be adversely affected. Overall, vertical integration is welfare-reducing because the negative effect of integration is large enough to outweigh any positive effect. Therefore, individually optimal decision on data disclosure might not be socially optimal when aggregated. The empirical analysis based on Google Android Application market data shows that a consumer, in general, has a privacy concern when making a decision. It also confirms the asymmetric privacy concern with respect to firm reputation, which means that consumers are more reluctant to provide their information to relatively small app providers with no great market reputation. The empirical findings, therefore, corroborate the hypotheses in the model.

}
		
\end{singlespace}
\bigskip
\begin{small}\textbf{Keywords} \quad Privacy, Information Acquisition, Data Intermediary, Targeted Ad,  Vertical Integration \\
\textbf{JEL Codes} \,\, D21; D22; D83; L15; L22; L42; L52\end{small}
%\pagebreak 
%\tableofcontents
%\pagebreak


\section{Introduction}



It is a well-known fact that platforms such as Facebook or Cable TV operators play a role as a data intermediary by gathering and selling user's personal information for revenue generating purposes: selling the data directly to third parties or using the data to deliver more targeted advertising(ad) to consumers. For example, Facebook requires any user to provide some mandatory information including name, gender, email address, and birthday to register with the site. On top of that, some users provide additional information such as education, work, their likes, and interests. Facebook monetizes such detailed demographic information by providing targeted ad opportunities for various sellers. Obviously, the platform makes more money as more user information becomes available. From the more available personal information, each seller or data holder who purchases the data from Facebook could make more attractive targeted ads. In that sense, each consumer's privacy concern which determines the total amount of personal data available plays an important role in the platform's as well as each seller or advertiser's businesses. In other words, since the amount of information available affects each seller's overall targeting effectiveness, an individually optimal decision on how much information to disclose in on-line networks affects seller market competition where each seller attracts consumers by making targeted ads based on personal information.

As seller's overall targeted ads become more effective, consumers would face lower mismatch cost from that seller in a sense that targeted ads suggest more suitable products to their current needs. Interestingly, this potential benefit from a privacy loss is likely to have an asymmetric effect on different sellers in terms of the market competitive structure. That is because that a consumer is likely to face much higher mismatch cost from small sellers or market entrants whose initial targeting skill is worse than market incumbents: the incumbents have better initial targeting technology which has been developed from previous sales experience or existing customer data whereas the entrants who do not have such experience have worse targeting skill, to begin with. All other things equal, consumers are more likely to choose better targeting sellers because they could save more mismatch cost.  Given this asymmetry, there are some interesting research questions. How does the total amount of personal information available affect market competition where the entrant with worse targeting skill demands personal data a lot more than the incumbent whose targeting skill is good enough even with small data set? What if the only limited amount of personal information becomes available due to more privacy concern? Does it have more adverse effect on market entrant? How does it affect social welfare ultimately? 

To answer those questions, I set up a model in which asymmetric sellers with respect to initial targeting technology, as well as overall product quality, decide whether to purchase consumer data from the platform and engage in price competition later on. In the model, a consumer is characterized by two-dimensional type with respect to privacy sensitivity and the valuation of product quality. Depending on his privacy type, each consumer decides whether to disclose personal information or not by comparing a trade-off between potential benefit arising from better-targeted ads (lower mismatch cost) and nuisance cost from a loss of privacy. The main result shows that the entrant with worse targeting skill always wants to buy data from the platform whereas the incumbent buys only when the less amount of personal data becomes available due to more privacy concern. Intuitively, if only limited information is available, the entrant's targeting is still bad enough, even after obtaining data, for the incumbent to easily beat the rival by purchasing the same set of data as well. However, in equilibrium, the portion of consumers who discloses information to enjoy better-targeted ads is sufficiently large enough for the platform to sell data exclusively to the entrant at a higher price. Due to the high data price, the entrant might not earn more profit from data monopolization although he earns greater market share and revenue. In other words, the entrant ends up choosing the suboptimal choice which is to buy data at a higher price when the incumbent does not buy. 


In addition to the benchmark model described above, this paper makes a specific extension of data-driven vertical integration. Vertical integration between platforms and any seller has become major concerns because they bring about other anti-competitive problems regarding data sharing policy. Most of the data-driven vertical integration is likely to foreclose downstream rivals from data access, which brings about another antitrust threat. Recently, AT\&T and Time Warner have announced the merger plan. If this merger is approved, the integrated firm would have the extensive amount of collective customer data. By using the data exclusively, Time Warner can target consumers much more effectively, thereby attracting them more. Since this gives an unfair advantage to the integrated downstream firm, it leads to antitrust issues. Indeed, the result from vertical integration model in this paper shows that data-driven vertical integration always leads to data foreclosure and whether the platform vertically integrates with either the incumbent or entrant depends on the portion of privacy-sensitive consumers: if consumers are more likely to be privacy-sensitive so that less amount of personal information becomes available, the platform and incumbent vertically integrate and foreclose the entrant from data access. In this case, the entrant suffers from lower market share, thereby obtaining lower profit due to data foreclosure. However, the integration with entrant and resulting data foreclosure do not adversely affect unaffiliated incumbent since he would voluntarily choose not to buy data even without vertical integration for the parametric range which leads to vertical integration with entrant in equilibrium. The results imply that if there are more privacy-sensitive consumers, the integration can create anti-competitive effect toward the entrant. Eventually, consumers might be also harmed from lack of competition; individually optimal decision on data disclosure might not be socially optimal when aggregated.

From the welfare analysis result, vertical integration with the incumbent always makes consumers worse off compared to that with the entrant or no vertical integration. Though the integrated firm enjoys much higher profit under vertical integration, total social welfare is also lower since the negative effect on consumer is large enough to outweigh any benefit for integrated firm's profit. 

I also discuss the effect of seller's reputation on the willingness to disclose information and the resulting market competition. It is likely that consumers are more reluctant to provide their information to unknown seller due to greater concern about possible data abuse. Given that the incumbent has much greater market reputation than the entrant, this asymmetric privacy concern with respect to firm's reputation might more adversely affect the entrant's business. The modified model confirms that consumer's asymmetric privacy concern aggravates the entrant's disadvantage, thereby making him worse off. 

Last, I provide an empirical evidence by using Google Android App Market data to show that the implicit hypotheses I've made throughout the paper comply with reality. Specifically, the evidence corroborates that consumers care about privacy when choosing a product and their privacy concern is pretty much affected by data collector's reputation. 
\\\\
\textbf{Previous Literature} \quad 
Driven by the rise of the Internet and Big Data, many recent literatures examine how firm's ability to sell customer information which is related to privacy concern affects the relevant market structure. Acquisti and Varian (2005), Conitzer et al. (2012), Belleflamme and Vergote (2016), and Koh et al. (2017) allow consumers to actively decide how much information to disclose.  However, those papers assume the monopolistic seller so do not examine how privacy concern or information availability affects market competition

%\footnote{e.g. See Taylor 2004; Kim and Wagman 2015. Those papers, however, assume passive consumers who do not actively decide whether to protect their privacy. Taylor (2004) divides naive and sophisticated consumer types but analyze two cases separately by assuming that types are exogenously given. For foundational literature, see Posner, 1981; Stigler, 1980; Fudenberg and Tirole, 2000; Fudenberg and Villas-Boas, 2012; and Villas-Boas, 1999, 2004. However, those papers just focus on how information disclosure about consumer's purchasing history or private valuation affects pricing structure or dynamic price discrimination strategy but do not directly take into account consumer privacy. For a detailed and recent survey, see Acquisti et al. (2016).} .\footnote{Though Acquisti and Varian (2005) analyze how conditioning prices on purchase history affects market competition as one of the generalizations, their focus is how firms can find it profitable by locking-in consumers who face switching cost.} 

The closest previous literature to my work is research about the effect of privacy on market competitive structure. Taylor and Wagman (2014) examine how privacy enforcement leads to different competitive market outcomes depending on individual context and industries. Shy and Stenbacka (2016) suggest that there is non-monotonic relationship between the degree of privacy protection and equilibrium profits. While Casadesus-Masanell and Hervas-Drane (2015) also study the similar issues, they model such that consumers decide how much their information to provide as similarly in Koh et al. (2017). Also, Montes et al. (2016) also endogenize privacy by allowing consumers to anonymize themselves with a cost and analyzes how privacy concern and the resulting information availability affect competing firms' price discrimination and data acquisition decision, consumer surplus, and social welfare. However, those papers do not consider asymmetric sellers, thereby not seeing any disproportionate effect toward market entrants. In that sense, Campbell et al. (2015) which shows that small firms or entrants can be adversely affected by privacy regulation which imposes unit costs to all firms shares one of the main implications of my paper. However, my paper still differs from theirs in a sense that I adopt a few more asymmetries in seller's market in terms of initial targeting technology and product quality as well as heterogeneous consumer privacy sensitivity. By having such asymmetries and heterogeneity, I offer a microfoundation of how consumers react differently to potential privacy risk and how sellers are disproportionately affected by privacy concern. Though Braulin and Valletti (2016) consider vertically differentiated sellers as my paper to see how exclusive data sales affects consumer and social welfare, they do not see any possible anti-competitive effect toward any seller without customer information. On top of that, none of those papers takes into consideration the data-driven vertical integration.\footnote{With the exception of Montes et al (2016) and Braulin and Valletti (2016); they mention exclusive data selling but do not see any anti-competitive threat of data foreclosure.} 


Last, my paper adds the empirical evidence which corroborates the implicit assumptions and hypotheses which I made throughout the paper. The most relevant work to my paper is Kummer and Schulte (2016) who study a money-for-privacy trade-off in the smartphone applications market. In the paper, they provide an empirical evidence that both the market's supply and demand side take into consideration an ability of apps to gather personal information when making a decision.

In summary, my paper contributes to the literature for a few unique reasons. First, by allowing consumer heterogeneity and seller market asymmetry together, the model can answer much more realistic and interesting questions such as how consumer's privacy concern and the consequential information availability affects competition with asymmetric sellers. More importantly, I analyze the data-driven vertical integration between a data collector and advertisers. While there are a number of papers which study about vertical integration issues in on-line market, most studies just examine the effect of vertical integration on pricing scheme and the relevant market competitive structure but do not focus on data-sharing aspects.  


	\section{Model}
Let $\Gamma=(N, S, u)$ be a finite normal form game where $N=\{Consumer, Platform, Seller A, Seller B\}$ is the set of players, $S=\cup_{n\in N}S_n$ is the set of strategy profiles, and $u: S \rightarrow \mathbb{R}_+$ is the payoff function. 


\textbf{Strategy} \quad There is a unit mass continuum of consumers indexed by $i \in [0,1]\times[0,1]$. Each consumer $i \in [0,1]\times[0,1]$ has two-dimensional type $\rho_i=(\tau_i, \theta_i) \in P \equiv T \times \Theta$. First, $\tau_i$ denotes each consumer's privacy sensitivity which is horizontally distributed over $[0,1]$ with uniform distribution function. Consumer $i$ is more privacy-sensitive as $\tau_i$ increases. Second, $\theta_i$ denotes consumer $i$'s valuation with respect to product quality sellers provide which is vertically distributed by $U \sim[0,1]$. $\tau_i$ and $\theta_i$ are independently distributed of each other. Depending on each type $(\tau_i,\theta_i)$, each consumer makes two independent decisions; (a) whether to disclose his personal information to the platform or not ($a_c^{info} \in \{D, ND\}$), and (b) whether to purchase a product from high or low quality seller ($a_c^{purchasing} \in \{H, L\}$). This binary choice setup implies that both of $\tau$ and $\theta$ have cutoffs which determine consumer $i$'s choices. Let $\mathcal{IS}$ denote the set of privacy-insensitive consumers whose privacy sensitivity $\tau_i$ is smaller than a certain threshold called $\tau$. The rest of privacy-sensitive consumers whose $\tau_i$ is larger than $\tau$ are in the set of $\mathcal{S}$. In other words, $P(i \in \mathcal{IS})=\tau \in(0,1)$ and $P(i \in \mathcal{S})=1-\tau$. If a consumer $i \in \mathcal{IS}$, he provides as much personal information as possible to the platform (chooses ``$D$''). If $i \in \mathcal{S}$, he does not provide any of his private information (chooses ``$ND$''). This means that $P(i \, \text{chooses}\, D|i\in\mathcal{IS})=1$ and $P(i \, \text{chooses}\, D|i\in\mathcal{S})=0$. Similarly, let $\mathcal{H}$ denote the set of relatively high valuation consumers who buy from high-quality seller. The rest of low valuation consumers is in the set of $\mathcal{L}$. If the threshold on $\theta$ which decides $\mathcal{H}$ or $\mathcal{L}$ is given by $\bar{\theta}$, $P(i \in \mathcal{H}) = 1-\bar{\theta}$ and $P(i \in \mathcal{L})=\bar{\theta}$. Thus, the strategy of consumers is a mapping from type $(\tau_i,\theta_i)$ to information disclosure and purchasing decisions; $S_c:[0,1]\times[0,1] \rightarrow \{\mathcal{IS} \cap \mathcal{H}, \mathcal{IS} \cap \mathcal{L}, \mathcal{S} \cap \mathcal{H}, \mathcal{S} \cap \mathcal{L}\}$. For example, any consumer $i \in \mathcal{IS}\cap\mathcal{H}$ chooses an action of $D$ and $H$. 

The platform market is monopolistic. The platform gathers personal information about customers while providing diverse services to them. The amount of data available depends on how likely each consumer discloses his information to the platform, i.e., whether a consumer $i \in \mathcal{IS}$ or $i \in \mathcal{S}$. Though both types of consumers provide basic information to the platform to enjoy services it offers, the platform only sells detailed information. Thus, the maximum amount of information available for selling can be denoted as $\bar{d}\equiv d\tau$ where $d$ denotes the total amount of detailed demographic information which the platform asks each consumer such as economic and family status; if there are more privacy-insensitive consumers (higher $\tau$), $\bar{d}$ increases. Normalizing that $d=1$, the platform sells $\bar{d}=\tau$ amount of information to any seller who wants to buy. 

Each seller ($A$ or $B$) sells a variety set of products to consumer, i.e., plays a role as a retailer. The set of products from each seller is vertically differentiated in terms of quality denoted by $s_j$.\footnote{For example, both of \textit{Mercedes-Benz} and \textit{Honda} have very similar line-up of vehicles (coupes, sedan, or SUV) but the overall quality is different.} To begin with, both of sellers have very basic information about all consumers (denoted by $\underline{d}$) such as email address, gender, or date of birth. On top of that, sellers might purchase more detailed information about some consumers from the platform. This data acquisition decision depends on data price $C$ charged by the platform and his initial targeting skill denoted by $\gamma_j$. Based on the information he has, each seller makes targeted ads to consumers. The overall quality of targeting depends on (a) seller $j$'s initial targeting technology, $\gamma_j$, and (b) total amount of personal information about potential customers denoted by $D_j$. Normalizing the basic information $\underline{d}$ to one, $D_j$ can be simply denoted as either $1+ \tau \equiv \bar{D}>1$ if seller $j$ buys data from the platform or one otherwise. The overall quality of targeting increases in $\gamma_j$ as well as $D_j$. To sum up, each seller is asymmetric along two dimensions; overall product quality and initial targeting technology. Without loss of generality, I first assume that $\gamma_A >\gamma_B$: Seller $A$ has better targeting technology than seller $B$. I can also interpret this asymmetry such that seller $A$ is an incumbent who has previous sales experience and existing customer information whereas seller $B$ is an entrant without such experience. As for product quality, either $A$ or $B$ can provide a high-quality product. I focus on the case in which $A$ is better in initial targeting but $B$ is better in product quality, thus, $s_A<s_B$ is assumed throughout the paper. For example, Amazon can be a representative example of the incumbent firm. Given that Amazon sells various products in many categories, it is a generalist. Any specialist on-line retailers which sell various products in a specific category, such as apparel, can be considered as high-quality entrants in a sense that they have specialty in their own business area. 


\textbf{Payoff} \quad First, the platform only earns his profit from selling user data to any seller. By optimally setting per unit data price $C$, the platform realizes his profit maximization problem as follows. 
\begin{align}\label{platform_profit}
\max_{C} \quad \pi_p(C|n, \tau) = n\tau C
\end{align}
where $n$ denotes the number of sellers who buy the data and the subscript $p$ denotes the \textit{platform}. As for each seller $j$, the profit maximization problem for each seller $j$ is defined as follows. 
\begin{align}\label{seller_profit}
\max_{P_j, D_j} \quad \pi_j = P_j X_j(P_j,D_j|\gamma, s) - \mathbbm{1}_{\text{buy}}  C \tau
\end{align}
where $P_j$ is price that seller $j$ charges to consumers and $X_j(P_j,D_j|\gamma, s)$ is $j$'s aggregated market share. If $j$ buys data from the platform, he needs to pay the price of $C$ set by the platform. 

Lastly, I abstract from consumer's utility obtaining from using platform's services. Rather, each consumer obtains utility from purchasing a product from a seller $j$. Assuming that sellers are able to make more effective targeted ads from personal data, any $i \in \mathcal{S}$ is likely to suffer from higher \textit{mismatch cost} in a sense that targeted ads suggest more suitable products to consumers. Here, the mismatch cost captures a potential benefit from a loss of privacy. In addition, a consumer also faces a potential nuisance cost from privacy loss. The privacy nuisance cost increases in privacy sensitivity $\tau_i$. The utility taking into consideration this mismatch cost for each consumer $i$ from seller $j$ is given as follows.
\begin{align}\label{consumer_utility_endogenous}
\begin{aligned}
u_{ij}= \begin{dcases}
&\,\,  V+\theta_i s_j - P_j -  \frac{1}{\gamma_j D_j}-\psi(\tau_i) \quad \quad  \text{if disclosing information ($i \in \mathcal{IS}$)}  \\
&\,\,  V+\theta_i s_j - P_j -  \frac{\alpha}{\gamma_j D_j} \quad\quad\quad\quad\quad\,\,  \text{otherwise ($i \in \mathcal{S}$)} 
\end{dcases}\end{aligned}
\end{align}
where $\psi(\tau_i)$ denotes the nuisance cost with $\psi'>0$, $\psi'' \geq 0$ and $\tau_i \sim U[0, 1]$. I also assume that $\psi(\tau_i)$ is invertible. $V$ denotes the reservation value (base utility) which is assumed to be large enough to fully cover the market and the valuation of consumer $i$ with respect to product quality is given by $\theta_i  \sim U[0, 1]$. $s_j$ is the overall product quality and $P_j$ denotes the average price of products from seller $j$. $ \frac{\alpha_i}{\gamma_jD_j}$ is the mismatch cost where $\alpha_i=1$ if $i \in \mathcal{IS}$ and $\alpha_i=\alpha>1$ if $i\in \mathcal{S}$. That is, if consumer $i$ is privacy-insensitive so provides as much personal information as possible, he is likely to receive more suitable ads from any seller. In that sense, privacy-sensitive consumers who provide no detailed information suffer from higher mismatch cost. The mismatch cost also depends on seller $j$'s targeting skill ($\gamma_j$) and how much detailed data each seller has ($D_j$). Given that $\gamma_A>\gamma_B$, a consumer is more likely to incur lower mismatch cost from seller $A$ whose targeting skill is better. Last, I assume that a consumer faces lower mismatch cost as seller $j$ has more consumer information when making targeted ads and the effect of data on saving mismatch cost is non-increasing as $D_j$ increases. For the simplicity, I normalize $\gamma_B$ to one and denote $\gamma_A$ as $\gamma$.


 Here is more intuitive explanation about the mismatch cost. A consumer knows that there are two sellers (retailers) $A$ and $B$ whose set of products are identical but the overall quality of products is different. A consumer does not have a preference for a specific brand(seller) over the other but has different needs for a specific type of products which both sellers sell. The needs for a specific type of products can change over time depending on diverse circumstances such as change in marital status. Thus, if each seller has the most recent information about potential customers, he is able to make more attractive targeted ads to consumers. Then, consumers could spend less time on finding the most suitable product for himself because the consumer obtains the relevant information immediately. Thus, the mismatch cost does not arise from comparing one seller to another but from comparing types of products which each retailer sells: if one seller suggests the most suitable product which fits my current need, the suggestion would be a good match for my need.
 %\footnote{For one thing, as similarly in Levin and Milgrom(2010), suppose that a male subscriber in his 20s recently had a baby so is planning to buy a minivan for his family. AT\&T as a cable TV operator has privacy-sensitive information about subscribers' viewing patterns which says that he regularly watches \textit{SpongeBob} series rather than \textit{The Walking Dead}. Assume that \textit{Honda} purchases this additional information about him while \textit{Subaru} does not. Then, \textit{Honda} would make a minivan ad but \textit{Subaru} would make a sports car ad from the guess that men in their 20s prefer sports car in general. Then, it is more likely for the consumer to go to \textit{Honda} than \textit{Subaru} because when a consumer goes to \textit{Honda}, he knows exactly what features to look for in the car type he wants, whereas at \textit{Subaru}, he still has to look up additional details. Thus, even if he knows the price of the \textit{Subaru} and \textit{Honda}, so search is sunk, the transaction is cheaper at \textit{Honda}.} 
 
 In this specification, consumers who do not provide any additional personal information to the platform also benefit as seller $j$ obtains more amount of aggregate information from the platform. This assumption implies that privacy-sensitive consumers would receive some amount of targeted ads in a form of promotion emails, for example. This is plausible scenario considering \textit{information externality}. Given that some basic information about consumers is available, firms can predict information about consumers who refuse to provide their personal information based on the information obtained from other consumers who provide some personal data in the network. For example, firms can categorize consumers into some subgroups based on gender and age. Among each consumer category, some people provide much information about themselves while others provide nothing. Then, the information can be passed along to peer group, so that consumers who do not provide any further personal information are likely to receive some promotion emails.\footnote{See Choi et al (2016) for more detailed description on information externality.}
 
 
\textbf{Timing and Equilibrium Concept} \quad Every information except for a true $\tau_i$ is a common knowledge. The platform and each seller know the distribution of $\tau_i$ as $\tau_i \sim U[0, 1]$. I investigate whether and how data-driven vertical integration between the platform and one seller is made in order to examine any antitrust implication. As a benchmark model, I also analyze no vertical integration case. In both games, the platform and each seller form beliefs about consumers' privacy sensitivity given their identification status ($\mathcal{IS}$ for privacy-insensitive and $\mathcal{S}$ for privacy-sensitive). Thus, the solution concept I use for this game is Perfect Bayesian Nash Equilibrium (PBE). 
 
In no vertical integration, the timing of game is as follows. $S_n^t$ denotes strategy profile for player $n \in N$ in timing $t$. 
\begin{enumerate}
	\item Consumer $i$ decides whether to disclose his personal information to the platform depending on $\tau_i \sim U[0,1]$. ($S_c^1: [0,1] \rightarrow a_c^{info} \in \{D, ND\}$) 
	\item  After observing consumer's identification status ($\mathcal{IS}$ or $\mathcal{S}$), the platform aggregates all available information from $i \in \mathcal{IS}$ and forms a belief, $\mu(a_c^{info})$, about consumer's privacy type. He also sets an optimal data price $C$ by maximizing his profit defined as in (\ref{platform_profit}). ($S_p^2: \mu(a_c^{info}) \rightarrow C \in \mathbb{R}_+$)
	\item Each seller with asymmetric targeting technology $\gamma_j$ first decides whether to purchase data from the platform after observing the price $C$. ($S_j^{3(i)}:(C, \gamma_j)\rightarrow D_j \in \{1, \bar{D}\}$) After that, seller $j$ sets an optimal product price $P_j$. ($S_j^{3(ii)}:(D_j,D_{-j},s_j) \rightarrow P_j \in \mathbb{R}_+$ where $D_{-j}$ denotes the rival's data acquisition decision) Each seller then makes a targeted ad by using the data he has and initial targeting technology.
	\item  Observing each seller's price and targeted ads, a consumer decides from which seller to purchase a product depending on his valuation with respect to product quality $\theta_i \sim U[0,1]$. ($S_c^4: ([0,1], P_j) \rightarrow a_c^{purchasing} \in \{H, L\}$ where $H$ denotes buying from $B$)\footnote{I focus on $s_B>s_A$ case here.} Each consumer obtains net utility by (\ref{consumer_utility_endogenous}). 
\end{enumerate}

In vertical integration game, I add one more stage in the beginning of second stage that the platform decides with whom to vertically integrate. Thus, the platform's strategy set has one more decision on the merger partner. After that, the timing of game is the same as above except that in the second stage affiliated seller always chooses $D_j = \bar{D}$ while unaffiliated seller still has two choices either $\bar{D}$ or 1. 

Therefore, a PBE consists of a strategy profile $(S_c^1, S_p^2, S_A^3, S_B^3, S_c^4)$ and the firms' beliefs about consumers' privacy sensitivity type $\mu(a_c^{info})$. These constitute a PBE if all strategies are sequentially rational given the beliefs and the beliefs are consistent given the strategies.


	\section{No Vertical Integration - Benchmark}
In the first stage, each consumer compares two different utility levels as in (\ref{consumer_utility_endogenous}) and decides whether to disclose information or not depending on $\tau_i$. Thus, by the time firms make their optimal decisions, they form beliefs on the proportion of consumers who disclose information, i.e. $P(i \in \mathcal{IS})$, from the distribution of $\tau_i \sim U[0,1]$. The beliefs are given by 
\begin{align}\label{beliefs}
\begin{aligned}
P(i \in \mathcal{IS}) &= P(i \in \mathcal{L}) P(i \in \mathcal{IS}|i \in \mathcal{L})+P(i \in \mathcal{H})P(i \in \mathcal{IS}|i \in \mathcal{H})\\
P(i \in \mathcal{S}) &= P(i \in \mathcal{L}) P(i \in \mathcal{S}|i \in \mathcal{L})+P(i \in \mathcal{H})P(i \in \mathcal{S}|i \in \mathcal{H})
\end{aligned}
\end{align}

By the time each seller and the platform make decisions, they treat $P(i \in \mathcal{IS})$ and $P(i \in \mathcal{S})$ as fixed. In other words, firms take those probabilities as given by $P(i \in \mathcal{IS}) = \tau$ and $P(i \in \mathcal{S})=1-\tau$. 

Given the beliefs, I solve for PBE using backward induction to obtain sequentially rational strategies. Since the nuisance cost of $\psi(\tau_i)$ is not a function of seller $j$'s choice, the marginal consumer type on valuation $\bar{\theta}$ who is indifferent between purchasing a product from low-quality $A$ and high-quality $B$ does not depend on $\tau_i$. From the utility specification in~(\ref{consumer_utility_endogenous}), the indifference condition is $\bar{\theta}_{\mathcal{S}} = \frac{P_B-P_A+\alpha(\frac{1}{D_B}-\frac{1}{\gamma D_A})}{s_B-s_A}$  for $i \in \mathcal{S}$. As for $i \in \mathcal{IS}$, the indifference condition is $\bar{\theta}_{\mathcal{IS}} = \frac{P_B-P_A+(\frac{1}{D_B}-\frac{1}{\gamma D_A})}{s_B-s_A}$. The weighted indifference condition can be rewritten in a simple way as follows.
\begin{align}\label{marginal theta}
\bar{\theta} = \dfrac{P_B-P_A+\Delta\tilde{\alpha}}{ s}
\end{align} 
where  $\Delta  =\frac{1}{D_B}-\frac{1}{\gamma D_A}$, $\tilde{\alpha}=\tau+\alpha(1-\tau)$, and  $s = s_B-s_A$. Let's assume $\bar{D}<\gamma$, which guarantees that $\Delta>0$. The market share for each seller is given by $ X_A = \bar{\theta}$ and $X_B =1-\bar{\theta}$ under $s_A<s_B$.

 Given $X_A $ and $X_B $, the solutions to profit maximization problem with respect to $P_j$ are given by 
\begin{align}\label{eqmpriceshare}
\begin{aligned}
&P_A = \dfrac{s+\Delta \tilde{\alpha}}{3}; \quad P_B = \dfrac{ 2s-\Delta\tilde{\alpha}}{3}; \quad X_A = \dfrac{s +\Delta \tilde{\alpha}}{3s}; \quad X_B= \dfrac{ 2s -\Delta \tilde{\alpha}}{3s}
\end{aligned}
\end{align}

To guarantee an interior solution, I assume that $\frac{\Delta\tilde{\alpha}}{2}<s$ throughout the paper; seller $B$'s quality, $s_B$, is large enough to have positive demand. 

Given the equilibrium price and quantity in~(\ref{eqmpriceshare}), each seller decides whether to purchase data from the platform. The gap between mismatch costs from two sellers, $\Delta$, is different depending on each seller's choices on $D_j$ either $\bar{D}=1+\tau$ (if purchasing) or 1 (otherwise) which can be derived as follows. 
\begin{footnotesize}\begin{align}\label{delta}
	\begin{aligned}
	\Delta = \begin{dcases}
	\frac{1}{\bar{D}}(1-\frac{1}{\gamma}) \equiv \Delta_{BB}  & \text{if both sellers buy data}  \\
	1-\frac{1}{\gamma\bar{D}} \equiv \Delta_{BN}&  \text{if only seller A buys data} \\
	\frac{1}{\bar{D}}-\frac{1}{\gamma} \equiv \Delta_{NB}  &\text{if only seller B buys data}  \\
	1-\frac{1}{\gamma} \equiv \Delta_{NN}  &\text{if both do not buy data}  \\
	\end{dcases}\end{aligned}
	\end{align}\end{footnotesize}

I can rank different $\Delta$ as $\Delta_{NB}<\Delta_{BB}<\Delta_{NN}<\Delta_{BN}$. Using (\ref{eqmpriceshare}) and (\ref{delta}), each seller's equilibrium profit level is realized. By comparing profits under two choices, I can derive thresholds on $C$ which guarantees for one seller to buy data given the rival's decision. The thresholds are 

%the conditional equilibrium profit for each seller is given by the Table~\ref{eqmprofit} below where $B$ denotes \textit{Buy}, $N$ denotes \textit{Not Buy}, and the first component is the profit for seller $A$ while the second is for seller $B$. 
%\begin{table}[h]
%	\begin{center}
%		\caption {Equilibrium Profits for Each Seller} \label{eqmprofit} 
%		\begin{tabular}{ r|c|c }
%			\cline{1-3}
%			\hline
%			\multicolumn{1}{r}{} & B & N\\ 
%			\hline \centering
%			B               &  $\frac{(s +\Delta_{BB} \tilde{\alpha})^2}{9s}-C\tau$, $\frac{(2s -\Delta_{BB} \tilde{\alpha})^2}{9s}-C\tau$ &$\frac{(s +\Delta_{BN} \tilde{\alpha})^2}{9s}-C\tau$,  $\frac{(2s -\Delta_{BN} \tilde{\alpha})^2}{9s}$ \\ \hline
%			N              &$\frac{(s +\Delta_{NB} \tilde{\alpha})^2}{9s}$, \quad $\frac{(2s -\Delta_{NB} \tilde{\alpha})^2}{9s}-C\tau$& $\frac{(s+\Delta_{NN} \tilde{\alpha})^2}{9s}$  ,\quad  $\frac{(2s-\Delta_{NN} \tilde{\alpha})^2}{9s}$ \\
%			\cline{1-3}
%		\end{tabular}
%	\end{center}
%\end{table}
%Each seller now compares his profits under two choices and make a decision simultaneously. By comparing profits, I can derive thresholds on $C$ which guarantees for one seller to buy data given the rival's decision. The thresholds are 

\begin{small}
	$\begin{dcases}
	&\text{Given $B$ buys, $A$ buys if} \quad C<\frac{\tilde{\alpha}(\Delta_{BB}-\Delta_{NB})(2s+\tilde{\alpha}(\Delta_{BB}+\Delta_{NB}))}{9s\tau} \equiv \bar{C}_A\\
	&\text{Given $B$ not buys, $A$ buys if}\quad C<\frac{\tilde{\alpha}(\Delta_{BN}-\Delta_{NN})(2s+\tilde{\alpha}(\Delta_{BN}+\Delta_{NN}))}{9s\tau} \equiv \bar{\bar{C}}_A\\
	&\text{Given $A$ buys, $B$ buys if}\quad C<\frac{\tilde{\alpha}(\Delta_{BN}-\Delta_{BB})(4s-\tilde{\alpha}(\Delta_{BN}+\Delta_{BB}))}{9s\tau} \equiv \bar{C}_B\\
	&\text{Given $A$ not buys, $B$ buys if}\quad C<\frac{\tilde{\alpha}(\Delta_{NN}-\Delta_{NB})(4s-\tilde{\alpha}(\Delta_{NN}+\Delta_{NB}))}{9s\tau} \equiv \bar{\bar{C}}_B\\
	\end{dcases}$
\end{small}
\bigskip

Unambiguously, $\bar{\bar{C}}_A>\bar{C}_A$ and $\bar{\bar{C}}_B>\bar{C}_B$: Given the rival does not buy, it is more likely for me to buy; data acquisition is strategic substitute since $\frac{\partial^2 \pi_j}{\partial D_A D_B} = \frac{-2\tilde{\alpha}^2}{9sD_A^2D_B^2\gamma}<0$. The intuition is the following. The data can be used to differentiate products in some senses because better-targeted ads from more available data are able to attract more consumers. Thus, consumer information which is used for better-targeted ads increases the product differentiation which softens price competition. 

Given $\bar{\bar{C}}_A>\bar{C}_A$ and $\bar{\bar{C}}_B>\bar{C}_B$, I can rank all four thresholds by assuming that all parameters ($s$, $\alpha$, and $\gamma$) are in reasonable ranges such as $[1,2]$. The assumption implies that the gap between product quality levels, initial targeting technology, and consumer-specific mismatch costs arising from privacy sensitivity are large but smaller than a certain level, which guarantees proper competition between two sellers with its own respective advantage. Since $A$ has the advantage on targeting technology with $\gamma>1$, sufficiently large $s$ implies that both of sellers have its own distinct advantages, one on targeting technology and the other on product quality. The upper bound also says that the respective advantage is not large enough to dominate the market. Throughout the paper, I assume that $\gamma=\alpha=2$ and $s \in [1,2]$ because such numerical assumptions do not change the results. This assumption leads to one specific ranking on $C$ as $\bar{C}_A<\bar{\bar{C}}_A<\bar{C}_B<\bar{\bar{C}}_B$. 
\begin{figure}\centering
	\includegraphics[width=120mm]{C_threshold2} 
	\caption{Thresholds on $C$}\label{thresholdsC}
\end{figure} 
From the ranking on $C$, the platform makes a data pricing decision. Given the equilibrium decision for each seller, the platform's profits under different level of $C$ are 
$\begin{dcases}
\pi_p^*(\bar{C}_A) = 2\bar{C}_A\tau \quad &\text{if} \quad C\leq\bar{C}_A, \quad \text{(Buy, Buy)}\\
\pi_p^*(\bar{\bar{C}}_B)=\bar{\bar{C}}_B\tau \quad &\text{if} \quad C\leq\bar{\bar{C}}_B,\quad \text{(Not Buy, Buy)}
\end{dcases}.$  
By comparing different profit levels, the platform sets the optimal $C$. The conditional set of equilibrium for each seller and the platform is summarized in the Lemma~\ref{NVeqm}. 
\begin{lemma}\label{NVeqm}
	There exists $\bar{\tau}$ such that:
	($i$) if $\tau <\bar{\tau}$, the platform sets $C^*=\bar{C}_A$ so both sellers buy data. Each seller sets price at $P_A^*=\frac{2 s (\tau +1)-\tau +2}{6 (\tau +1)}$ and $P_B^*=\frac{4 s (\tau +1)+\tau -2}{6 (\tau +1)}$. The market share for each seller is given by $X_A^*=\frac{2 s \tau +2 s-\tau +2}{6 s( \tau +1)}$ and $X_B^*=\frac{4 s (\tau +1)+\tau -2}{6 s (\tau +1)}$.
	($ii$) if $\tau >\bar{\tau}$, the platform sets $C^*=\bar{\bar{C}}_B$ so only seller $B$ buys data whereas seller $A$ does not. $P_A^*=\frac{2 s (\tau +1)+\tau ^2-3 \tau +2}{6  (\tau +1)}$ and $P_B^*=\frac{4 s (\tau +1)-(\tau -2) (\tau -1)}{6  (\tau +1)}$. The market share for each seller is given by $X_A^*=\frac{2 s \tau +2 s+\tau ^2-3 \tau +2}{6 s (\tau +1)}$ and $X_B^*=-\frac{-4 s (\tau +1)+\tau ^2-3 \tau +2}{6 s (\tau +1)}$.
	The corresponding threshold is $\bar{\tau} = 3+2 s-\sqrt{4 s (s+4)+1}$.
\end{lemma}
%The set of conditional equilibrium profits is given as follows. 
%\begin{align}\label{eqmprofitsNV}
%\begin{aligned}
%&\text{If} \quad \tau<\bar{\tau}\begin{dcases}
%&\,\, \pi_p^{BB} = -\frac{(\tau -2) \tau  \left(4 s (\tau +1)+(\tau -2)^2\right)}{18 s (\tau +1)^2}\\
%&\,\,  \pi_A^{BB}  =\frac{\left(2 s (\tau +1)+\tau ^2-3 \tau +2\right)^2}{36 s (\tau +1)^2} \\
%&\,\,  \pi_B^{BB}  =\frac{1}{36} \left(\frac{\left(\tau ^2-3 \tau +2\right)^2}{s (\tau +1)^2}+16 s+\frac{4 \left(\tau ^2-4\right)}{\tau +1}\right)\\
%\end{dcases} \\
%&\text{If} \quad \tau>\bar{\tau} \begin{dcases}
%&\,\, \pi_p^{NB}  = -\frac{(\tau -2) \tau  (4 s (\tau +1)+\tau -2)}{9 s (\tau +1)^2}\\
%&\,\,  \pi_A^{NB}  =\frac{\left(2 s (\tau +1)+\tau ^2-3 \tau +2\right)^2}{36 s (\tau +1)^2}\\
%&\,\,  \pi_B^{NB}  =\frac{(4 s+\tau -2)^2}{36 s}\\
%\end{dcases}
%\end{aligned}
%\end{align}

Anticipating firms' optimal decisions as above, each consumer makes a decision on information disclosure in the first stage. From the utility specification in (\ref{consumer_utility_endogenous}), the marginal privacy sensitivity for a consumer who is indifferent between disclosing and not disclosing can be derived as follows.
\begin{align}\label{disclosing prob}
\begin{aligned}
P(i \in \mathcal{IS}|i \in \mathcal{L}) & = P(u_{iA}^{Disclose}-u_{iA}^{Not}>0) \quad\quad\quad\quad\,\, \because \, \text{$A$ provides low-quality product} \\
& =P(\tau_i<\frac{\alpha-1}{\gamma D_A}) = \underbrace{\frac{1}{2 D_A}}_{\text{marginal $\tau$}} \quad\,\,\quad\quad \text{if}\,\,\psi(\tau_i)=\tau_i \sim U[0,1],\,\, \alpha=\gamma=2
\end{aligned}
\end{align} 

Similarly, $P(i \in \mathcal{IS}|i \in \mathcal{H})=\frac{1}{D_B}$. Substituting the equilibrium $P_j$ and $D_j$ as derived in the Lemma \ref{NVeqm} into $\bar{\theta}$, 
\begin{align}\label{disclosing_prob_final}
\begin{aligned}
P(i \in \mathcal{IS}) & = (\underbrace{\frac{s+(2-\tau)\Delta}{3s}}_{P(i\in\mathcal{L})=\bar{\theta}=X_A}) \big(\frac{1}{2 D_A}\big)+(\underbrace{1- \frac{s+(2-\tau)\Delta}{3s}}_{P(i\in\mathcal{H})=1-\bar{\theta}=X_B})  \big(\frac{1}{D_B}\big)\\
%P(i \in \mathcal{S}) & = (\frac{s+\tilde{\alpha}\Delta}{3s}) (1-F\big(\frac{\alpha-1}{\gamma D_A}\big))+(1- \frac{s+\tilde{\alpha}\Delta}{3s}) (1-F\big(\frac{\alpha-1}{D_B}\big))
\end{aligned}
\end{align} 

Applying rational expectations, a true $P(i \in \mathcal{IS})$ as in (\ref{disclosing_prob_final}) should be consistent with the firm's belief on it which is $\tau$; $P(i \in \mathcal{IS})=\tau^*$. The resulting equilibrium $\tau^*$ is summarized in the Lemma \ref{endogenous_tau} below. Note that $\tau_{D_AD_B}$ denotes the equilibrium $\tau^*$ under a specific data acquisition equilibrium where the subscript $D_j$ is either $B$(Buy) or $N$(Not Buy).
\begin{lemma}\textbf{(Endogenously Derived Equilibrium $\tau$)}\label{endogenous_tau}
	Under four different data acquisition cases, the equilibrium $\tau$ is implicitly determined by the following equations: (a) $\tau_{BB}$ from $\tau =\frac{10 s (\tau +1)+\tau -2}{12 s (\tau +1)^2}$, (b) $\tau_{BN}$ from $\tau =\frac{2 s (\tau +1) (4 \tau +5)+(\tau -2) (2 \tau +1)^2}{12 s (\tau +1)^2}$, (c) $\tau_{NB}$ from $\tau =\frac{2 s (\tau +1) (\tau +5)+(\tau -2) (\tau -1)^2}{12 s (\tau +1)^2}$, and (d) $\tau_{NN}$ from $\tau =\frac{10 s+\tau -2}{12 s}$. Among four $\tau$ levels, $\tau_{BB}$ is the largest whereas $\tau_{NN}$ is the smallest. The relative size of $\tau_{BN}$ and $\tau_{NB}$ depends on $s$.
\end{lemma}

 The resulting equilibrium $\tau$ under four different data acquisition cases is represented in the Figure \ref{tau_comparison}. Intuitively, when both sellers use data to make targeted ads, a consumer is more likely to face higher privacy nuisance cost. Therefore, $\tau_{BB}$ is the lowest. Likewise, if no seller uses detailed personal information, there is no nuisance cost, thus, $\tau_{NN}$ is the highest. When only one seller uses personal data, the relative size of $\tau$ depends on $s$. If $s$ is relatively small, more consumers are willing to provide information when seller $B$ uses data. If $s$ is larger than a certain threshold, however, there are more consumers who provide data when $A$ is the only one data holder. The intuition behind this asymmetry is the following. When $s$ is small, $X_A=\bar{\theta}=P(i\in \mathcal{L})$ is much greater than $X_B=1-\bar{\theta}=P(i\in \mathcal{H})$. From the equation (\ref{disclosing_prob_final}), the weight on $X_A$, $P(i \in \mathcal{IS}|i \in \mathcal{L})=\frac{\alpha-1}{\gamma D_A}$, has larger effect on the overall size of right-hand size for the case of $X_A>X_B$. Since (B,N) leads to lower $P(i \in \mathcal{IS}|i \in \mathcal{L})$ than that under (N,B), the right-hand side of (\ref{disclosing_prob_final}) is bigger under (N,B). Thus, $\tau_{BN}<\tau_{NB}$ if $s$ is sufficiently small. That is, for small $s$, more consumers demand low-quality $A$ and those consumers are highly susceptible to $A$'s data acquisition decision. Consequently, the overall portion of disclosing data is much lower when $A$ buys data. By the similar logic, if $s$ is relatively large, more consumers want high-quality $B$'s product. Then, the overall portion of disclosing population is more affected by $B$'s data acquisition, therefore, the resulting $\tau$ is larger under (B,N). 
			\begin{figure}[!tbp]
				\centering
		\includegraphics[width=110mm]{tau_comparison1}
	\caption{Equilibrium $\tau$ }\label{tau_comparison}
\end{figure}

Given the equilibrium $\tau$ in the Lemma \ref{endogenous_tau}, the no vertical integration equilibrium is as follows. From the Lemma \ref{NVeqm}, I've shown that either (B,B) or (N,B) arises in equilibrium depending on the threshold of $\bar{\tau}$. However, the lowest endogenous equilibrium $\tau_{BB}$ is always larger than $\bar{\tau}$, which implies that the equilibrium $\tau^*$ is always larger than $\bar{\tau}$ in any data acquisition case. This means that (N,B) is the only possible data acquisition equilibrium when there is no vertical integration. The next Proposition summarizes this final equilibrium.  
\begin{proposition}\textbf{(Endogenous Privacy NV Equilibrium)}\label{endogenous_no_VI_prop}
	In equilibrium, the portion of consumers who disclose information is large enough for the platform to sell data exclusively to seller $B$. Thus, (Not Buy, Buy) is the data acquisition equilibrium under no vertical integration game. The equilibrium $P_j$ and $C$ are defined as $P_j^*(\tau_{NB})$ and $C^*(\tau_{NB})$, respectively, where $P_j^*$ and $C^*$ are as in the Lemma \ref{NVeqm}. 
\end{proposition}

The Proposition \ref{endogenous_no_VI_prop} says that the data is exclusively sold to the entrant because the platform is able to extract the maximum rent from such exclusive selling. As pointed out in Montes et al. (2016), such data exclusive selling strategy accords with reality in which different firms are unlikely to obtain data on the same consumers though they do business in the same industry. 


\subsection{Implication}
The Lemma~\ref{NVeqm} says that if there are more privacy-sensitive consumers ($\tau<\bar{\tau}$), both sellers buy data while seller $B$ only buys if $\tau$ is larger than $\bar{\tau}$. By comparing two cases, it is easy to show that seller $B$ always suffers from lower marker share if seller $A$ also buys data. Thus, $B$ can enjoy higher market revenue if he is the only data holder. However, the profit effect is ambiguous because the seller needs to pay much higher price for data monopoly to the platform. By comparing $\pi_B^{BB}$ to $\pi_B^{NB}$ where superscript denotes data acquisition status, I can find another threshold on $\tau$ which determines that one of the profit levels is greater than the other. If $\tau > \frac{2-s}{1+s} \equiv\bar{\bar{\tau}}$, $\pi_B^{BB}>\pi_B^{NB}$ and the reverse holds under $\tau <\bar{\bar{\tau}}$. As in the Figure \ref{tau_comparison}, $\bar{\bar{\tau}}$ is always larger than $\bar{\tau}$, which might lead to ironic consequences. If $\tau$ is relatively small, (N,B) makes $B$ better off but this does not arise in equilibrium because $A$ always wants to buy data as well for small range of $\tau$. Since the best response for $B$ to $A$'s buying decision is also to buy, $B$ ends up choosing (B,B). Likewise, if $\tau$ is sufficiently large, (B,B) leads to higher profit for $B$ because the platform extracts too much rent for (N,B). However, for this range of $\tau$, $A$ always refuses to buy data, thereby leading to (N,B) in equilibrium. Therefore, except for the area in between two thresholds on $\tau$, the optimal choice for $B$ leads to suboptimal result in terms of profit. The Corollary \ref{NVtau} summarizes the finding.  

\begin{corollary}\label{NVtau}
Though (N,B) leads to greater market share and higher revenue for the entrant $B$, it does not necessarily make him better off due to data cost which the platform charges. Because of the high cost for data monopolization, (N,B) can lead to lower profit than (B,B) in most cases. If $\bar{\tau}<\tau<\bar{\bar{\tau}}$, the resulting equilibrium which is (N,B) leads to higher profit for $B$.
\end{corollary}

\section{Vertical Integration Case}
In this section, I analyze the effect of vertical integration between the platform and one of the sellers. As for the timing, after each consumer decides whether to disclose information or not, the platform first makes a vertical integration deal with one of the sellers. After the vertical integration deal was made, unaffiliated seller decides whether to purchase data from the platform. The affiliated seller always uses data for targeted ads. Later, each seller sets a price simultaneously, and then consumers decide. 

\subsection{Equilibrium}
By backward induction, each seller's price and market share are the same as before. Given them, I examine what happens if the platform makes a deal with one of the sellers. First, I assume that the platform merges with seller $A$ whose targeting technology is better. Given that seller $A$ always uses data, seller $B$ buys data if $C\leq\bar{C}_B$ but does not buy if $C>\bar{C}_B$. The profit for the integrated firm can be written as follows. 
\begin{align}
\begin{aligned}
\pi_{VA}=\begin{dcases}
&\,\,  P_A(\Delta_{BB})X_A(\Delta_{BB})+\bar{C}_B\tau\equiv \pi_{VA}(\Delta_{BB})  \quad \quad  \text{if seller $B$ buys }  \\
&\,\,  P_A(\Delta_{BN})X_A(\Delta_{BN}) \equiv \pi_{VA}(\Delta_{BN}) \,\,\,\quad\quad\quad\quad\,\,   \text{otherwise} 
\end{dcases}\end{aligned}
\end{align}
where the subscript $VA$ denotes \textit{Vertical Integration with $A$}. By comparing $\pi_{VA}(\Delta_{BB}) $ to $\pi_{VA}(\Delta_{BN}) $, the integrated firm decides whether to sell data to the unaffiliated seller by setting $C$.  
\begin{align}
\begin{aligned}
\pi_{VA}(\Delta_{BB}) -\pi_{VA}(\Delta_{BN})  &= -\frac{2 (\tau -2) \tau  (s+\tau -2)}{9 s (\tau +1)}\\
%&\Leftrightarrow \tau<2-s\\
%&\Leftrightarrow \quad s>\tilde{\alpha}(\bar{m}(1-2\gamma)+1)
\end{aligned}
\end{align}

Thus, if $\tau<2-s$, $\pi_{VA}(\Delta_{BB})<\pi_{VA}(\Delta_{BN})$; as $s$ increases, it is more likely to sell data to the rival. The intuition is as follows. If he forecloses the entrant from data access, the market is more easily dominated by the integrated firm. Nevertheless, if $s$ is sufficiently large, the data monopolization itself is not enough to attract more consumers because of the incumbent's lower product quality. Accordingly, the integrated firm rather earns higher profit from selling data. 

%$\frac{\tilde{\alpha}\overbrace{(\gamma-1)}^{(-)}\overbrace{(\gamma-3+\bar{m}(5\gamma-3))}^{(-)}}{2(2-\gamma)}>0$.

Now, I look into what happens if the platform merges with seller $B$. By the same logic above, the profit for the integrated firm and the difference between two profit levels are as follows. 
\begin{align}
\begin{aligned}
\pi_{VB}=\begin{dcases}
&\,\,  P_B(\Delta_{BB})X_B(\Delta_{BB})+\bar{C}_A\tau\equiv \pi_{VB}(\Delta_{BB})  \quad \quad  \text{if $A$ buys }  \\
&\,\,  P_B(\Delta_{NB})X_B(\Delta_{NB}) \equiv \pi_{VB}(\Delta_{NB}) \quad\quad\quad\quad\,\,\,\,   \text{otherwise} 
\end{dcases}\end{aligned}
\end{align}
\begin{align}\label{VBprofitcomparison}
\begin{aligned}
\pi_{VB}(\Delta_{BB})-\pi_{VB}(\Delta_{NB}) &= -\frac{(\tau -2) \tau  \left((\tau -2)^2-2 s (\tau +1)\right)}{18 s (\tau +1)^2}\\
%&\Leftrightarrow \quad s<\frac{\tilde{\alpha}(3-\bar{D})}{2\bar{D}} \equiv \tilde{s}\\
%&\Leftrightarrow \quad s<\tilde{\alpha}(\bar{m}(2-\gamma)-\gamma)\equiv \tilde{s}
\end{aligned}
\end{align}

Thus, if $\tau>s-\sqrt{s (s+6)}+2$, the integrated firm forecloses the data access. If the platform is integrated with seller $B$, it is more likely to foreclose the data access as $\tau$ increases, contrary to the former case. This is because that if $\tau$ is sufficiently large, seller $B$ is able to overcome his targeting disadvantage and dominate the market more easily. Therefore, the integrated firm rather sells data only if $\tau$ is small; data selling revenue effect is greater than market dominance effect from data foreclosure. Note that the right-hand side of inequality decreases in $s$. This means that if $s$ increases, it is more profitable to foreclose the data access because the combination of data monopolization and higher $s$ give more advantage toward the integrated firm. 

To see with whom the platform has an incentive to integrate, I compare profits for integration with $A$ to that with $B$. Since $s-\sqrt{s (s+6)}+2<2-s$ for $s\in[1,2]$, there are three possible cases; A-Foreclose or B-Sell for $\tau<s-\sqrt{s (s+6)}+2$, A-Foreclose or B-Foreclose for $s-\sqrt{s (s+6)}+2<\tau<2-s$, and A-Sell or B-Foreclose for $\tau>2-s$. First, if $\tau<s-\sqrt{s (s+6)}+2$, it is easy to show that $\pi_{VA}(\Delta_{BN})>\pi_{VB}(\Delta_{BB})$. Second, if $s-\sqrt{s (s+6)}+2<\tau<2-s$, there is another threshold on $\tau$ such that $\tau>\sqrt{(s-2) s+4}-s \equiv\tau^{VI}  $ leads to vertical integration with $B$ and data foreclosure. If $\tau<\tau^{VI}  $, the platform and $A$ integrate and foreclose the data access. Last, if $\tau>2-s$, the platform and $B$ integrate with each other and foreclose $A$ from data access in equilibrium. As a result, there are two possible cases in vertical integration equilibrium; (a) integration with $A$ if $\tau<\tau^{VI} $, and (b) integration with $B$ otherwise. In any case, there is no data selling equilibrium but data foreclosure always emerges in vertical integration game. 

On top of that, I need to check whether the platform and each seller have the incentive to vertically integrate with each other in the first place by comparing the joint profits of the platform and each seller under no vertical integration to the profit of the integrated firm. First, if the platform and $A$ are integrated, they always want to be integrated if (B,B) is the no vertical integration equilibrium. However, for the case of (N,B), vertical integration is more profitable only if $\tau<\frac{2}{3} \left(-s+\sqrt{(s-1) s+16}-1\right) \equiv \tau^{VA}$; if $\tau$ is sufficiently large, the platform can extract much higher revenue from $B$ by charging higher data price. Similarly, the platform and $B$ have an incentive to be integrated if $\tau>s-\sqrt{s (s+6)}+2 \equiv \tau^{VB}$. The intuition is similar as above that if $\tau$ is small, the platform is better off from selling data to both sellers at a lower price. Therefore, if $\tau^{VB}<\tau<\tau^{VA}$ holds, vertical integration can always happen. As in the Figure \ref{endogenous_VI}, vertical integration can emerge in the shaded area and the merger partner is determined by the threshold of $\tau^{VI}$. For the discussion's sake, I define two sets as follows: $\mathcal{A} := \{(s,\tau) \in \mathbb{R}^2|s\in[1,2], \tau^{VB}<\tau<\tau^{VI}\}$ for integration with $A$ and $\mathcal{B} := \{(s,\tau) \in  \mathbb{R}^2|s\in[1,2], \tau^{VI}<\tau<min\{\tau^{VA},1\}\}$ for integration with $B$. The Lemma~\ref{VIeqm} summarized this results. 


%\begin{figure}[!tbp]
%	\centering	\includegraphics[width=90mm]{VI_finaleqm}
%	%		\caption{Vertical Integration and Data Sharing}
%	\caption{Equilibrium under Vertical Integration}\label{VI_eqm}
%\end{figure}
\begin{lemma} \label{VIeqm}
Given that $\tau^{VB}<\tau<\tau^{VA}$, the platform has the incentive to vertically integrate with seller $A$ whose targeting technology is better if $(s,\tau) \in \mathcal{A}$ whereas the integration with $B$ emerges if $(s,\tau) \in \mathcal{B}$. Regardless of integration seller, the integrated firm sets the data price too high so the unaffiliated seller gives up buying data. Thus, (Buy, Not Buy)  or (Not Buy, Buy) might emerge in equilibrium with $C^*>\bar{C}_B$ or $C^*>\bar{C}_A$, respectively. 
\end{lemma}

From the Lemma \ref{endogenous_tau}, I can use the equilibrium $\tau_{BN}$ and $\tau_{NB}$ to finalize the vertical integration equilibrium. First, if $s$ is sufficiently small, both of $\tau_{BN}$ and $\tau_{NB}$ are smaller than $\tau_{VI}$. Therefore, for this range of $s$, $\tau$ is always lower than $\tau_{VI}$, which means that the integration with $A$ and (B,N) is the unique equilibrium. However, if $s$ is larger than a certain threshold, both of $\tau_{BN}$ and $\tau_{NB}$ are larger than $\tau_{VI}$. In this range of $s$, there are too many consumers who disclose information, thereby the integration with $B$ and (N,B) being the unique equilibrium (See the Figure \ref{endogenous_VI}). The final vertical integration equilibrium is summarized in the Proposition \ref{endogenous privacy VI eqm}. 
\begin{figure}[!tbp]
	\centering	\includegraphics[width=100mm]{endogenous_tau_VI}
	\caption{Endogenous $\tau$ Vertical Integration Equilibrium}\label{endogenous_VI}
\end{figure}

\begin{proposition}\label{endogenous privacy VI eqm} \textbf{(Endogenous Privacy VI Equilibrium)} 
	There exists a threshold on $s$, $\bar{s} \in [1,2]$ such that if $s<\bar{s}$, the platform and $A$ are integrated and (B,N) emerges in equilibrium whereas if $s>\bar{s}$, the platform and $B$ are integrated and (N,B) arises in equilibrium. The threshold $s$ is implicitly determined from $\tau_{BN}=\tau_{NB}$.
\end{proposition}

\subsection{Implication}

When there is no vertical integration, seller $B$ always wants to buy data to overcome his initial disadvantage on targeting skill but $A$ wants data only when $\tau$ is small. Considering that vertical integration always leads to data foreclosure, it is more likely to adversely affect the entrant $B$ who always needs the data access. To see how this affects sellers, especially seller $B$, I compare sellers' profits with and without vertical integration. First, if the platform and seller $B$ are integrated, unaffiliated seller $A$ becomes indifferent between with and without vertical integration. Under no vertical integration, seller $A$ ends up having the same profit level between buying and not buying data because the platform extracts all extra revenue in the form of data price. Since the vertical integration equilibrium is (N,B) which is the same as in one case of no vertical integration, $A$ has no difference in profit from the merger between the platform and $B$. Though there is no effect on profits, it is worth noting that data acquisition equilibrium of (N,B) is more likely to emerge under no vertical integration than under vertical integration. From the Proposition \ref{NVeqm}, (N,B) emerges if $\tau>\bar{\tau}$ under no vertical integration. If the platform and $B$ are integrated, it is shown that (N,B) arises if $\tau>\tau^{VI}$. Since $\bar{\tau}<\tau^{VI}$ for $s \in [1,2]$, no vertical integration is much easier for $B$ to have data monopolization in equilibrium. In other words, vertical integration allows $A$ to foreclose data access for $\bar{\tau}<\tau<\tau^{VI}$ which is the region that $B$ would enjoy higher revenue by monopolizing the data under no vertical integration.

If the platform and $A$ are integrated, unaffiliated seller $B$ always suffers from lower profits due to the integration and consequential data foreclosure; $\text{min}\{\pi^*_B(\Delta_{BB}),\pi^*_B(\Delta_{NB})\}>\pi_{VA}^{*B}(\Delta_{BN})$. Thus, due to the data foreclosure, seller $B$ suffers from lower profit under vertical integration in general. That is because seller $B$ has the lower market share under the integration with $A$ case than any case under no vertical integration which leads to either (B,B) or (N,B). The Proposition \ref{VIBsuffering} summarizes this implication. 

\begin{proposition}(\textbf{Anticompetitive Effect of Vertical Integration})\label{VIBsuffering}
When the platform is vertically integrated with seller $A$, seller $B$ suffers from lower market share, thereby obtaining lower profit due to data foreclosure. For the integration with $B$, seller $A$ faces no difference in profit with and without vertical integration. 
\end{proposition}

This result brings up very important antitrust implication regarding data-driven vertical integration. Though the vertical integration always forecloses the data access regardless of with which seller to make the deal, the entrant seller is more likely to be harmed. Given that consumer information is vital for a small seller or market entrant whose targeting skill is worse, the vertical integration with better targeting seller is likely to play an anti-competitive role because it forecloses the unaffiliated entrant from using data to overcome his initial disadvantage.


\section{Welfare Analysis}
Based on the equilibrium results derived so far, I examine welfare consequences in this section. First, total social welfare function is the sum of consumer surplus and sellers' profits including the platform's profit as follows.
\begin{align}
\begin{aligned}
&SW_{NV} = CS_{NV} + \pi_{NV}^p + \pi_{NV}^A+\pi_{NV}^B\\
&SW_{VA} = CS_{VA}+\pi^A_{VA}+\pi^B_{VA}\\
&SW_{VB} = CS_{VB}+\pi^A_{VB}+\pi^B_{VB}
\end{aligned}
\end{align}
where $CS$ denotes consumer surplus, subscript $VA$ ($VB$) denotes vertical integration with seller $A$ ($B$) while $NV$ denotes No vertical integration, $\pi^A_{VA}$ ($\pi^B_{VB}$) is the profit for the integrated firm, and $\pi^B_{VA}$ ($\pi^A_{VB}$) is that for non-integrated firm. The profits for firms under benchmark and vertical integration model are all given above. Consumer surplus can be obtained in the following way. 
\begin{align}
\begin{aligned}
CS& = \tau\big[\int_0^{\bar{\theta}_\tau} (V+\theta s_A - P_A -\frac{1}{\gamma D_A}-\tau)d\theta +\int_{\bar{\theta}_\tau}^1 (V+\theta s_B- P_B - \frac{1}{D_B}-\tau)d\theta\big]\\
&+ (1-\tau)\big[\int_0^{\bar{\theta}_{1-\tau}} (V+\theta s_A - P_A -\frac{\alpha}{\gamma D_A})d\theta +\int_{\bar{\theta}_{1-\tau}}^1 (V+\theta s_B- P_B - \frac{\alpha}{D_B})d\theta\big]
\end{aligned}
\end{align}

			\begin{figure}[!tbp]
	\centering
	\begin{minipage}[b]{0.45\textwidth}
		\includegraphics[width=\textwidth]{CS_tau_NB_BN2}
	\caption{$CS_{NV}^{NB}=CS_{VB}$ vs $CS_{VA}$ (solid for NV and dotted for VI) }\label{CS1}
	\end{minipage}
	\hfill
	\begin{minipage}[b]{0.45\textwidth}
		\includegraphics[width=\textwidth]{SW_tau_NB_BN2}
		%		\caption{Vertical Integration and Data Sharing}
		\caption{$SW_{NV}^{NB}=SW_{VB}$ vs $SW_{VA}$ (solid for NV and dotted for VI)}\label{SW1}
\end{minipage}\end{figure}

For simplicity, I stick to the previous assumptions and $V=2$. The main focus here is to see how data-driven vertical integration affects consumer surplus and total social welfare. As shown above, (N,B) is the only possible data acquisition equilibrium under no vertical integration. Under vertical integration, (B,N) emerges if $s<\bar{s}$ whereas (N,B) emerges if $s>\bar{s}$. Note that (N,B) with and without vertical integration leads to the same level of consumer surplus and social welfare. Thus, if $s>\bar{s}$, vertical integration does not have any impact on consumer surplus as well as social welfare. I, therefore, compare (N,B) case for no vertical integration equilibrium to (B,N) for vertical integration equilibrium assuming that $s<\bar{s}$. As in the Figure \ref{CS1} and \ref{SW1}, no vertical integration case where $B$ is the only data holder attains higher consumer surplus and social welfare level than vertical integration with $A$ where $A$ forecloses $B$ from data access. The intuition is as follows. $A$ charges higher price but enjoys greater market share than $B$ under (B,N). Under (N,B), the reverse holds. However, the price gap under (B,N) is much larger than that under (N,B) for the range of equilibrium $\tau$. In other words, the incumbent $A$ charges too high price if he monopolizes the data, which is harmful to consumers.  

\begin{proposition}\textbf{(Welfare Consequence of Vertical Integration)}\label{welfare}
Data-driven vertical integration with the incumbent is always welfare-reducing than either no vertical integration or vertical integration with the entrant. 
\end{proposition}

\section{Discussion on Asymmetric Privacy Concern}
So far, the model assumes that each consumer has symmetric privacy nuisance cost of $\psi(\tau_i)$ no matter who uses data. However, people are more likely to care about who asks for personal information; if unknown seller asks privacy-sensitive information, people are more reluctant to agree with that due to greater privacy concern. As the empirical evidence in next section will show, it is reasonable to assume that privacy concern is asymmetric with respect to data collector's reputation. Given that the incumbent has greater market reputation, I assume that any consumer who discloses information faces higher annoyance cost from the entrant $B$ than the incumbent $A$.  In other words, privacy nuisance cost in (\ref{consumer_utility_endogenous}) now also depends on seller $j$'s reputation such that $\psi_A(\tau_i)<\psi_B(\tau_i)$. For simplicity, I assume that $\psi_A(\tau_i)=\tau_i$ and $\psi_B(\tau_i)=\tau_i+b$ where $b$ is the additional nuisance cost arising from seller's lack of reputation.\footnote{The additive separable and fixed form of nuisance cost is for discussion's simplicity. However, the qualitative result should hold with any functional forms of nuisance costs as long as $\psi_A(\tau_i)<\psi_B(\tau_i)$.} 

In the asymmetric privacy cost model, $\tau_{NB}^{Rep}<\tau_{BN}^{Rep}$ regardless of the size of $s$ where superscript $Rep$ denotes the reputation privacy model with $b$. This implies that consumers are always less willing to disclose information to the entrant. Also, any equilibrium $\tau$ is always lower than that under the previous model without $b$ if there is at least one seller who buys data. Due to less willingness to disclose information, i.e., lower $\tau$ in equilibrium, (B,B) can emerge when there is no vertical integration. As shown in the Lemma \ref{NVeqm}, the entrant suffers from lower market share if the incumbent also buys data. That is, the entrant might become worse off from asymmetric privacy concern even without vertical integration and the consequential data foreclosure. Moreover, lower $\tau$ also affects vertical integration equilibrium in that the integration with $A$ is more likely to arise. Put differently, there is a new threshold,  $\tau_{VI}^{Rep}$, below which the integration with $A$ and data foreclosure emerge in equilibrium. Due to high privacy nuisance cost from $B$, $\tau_{VI}<\tau_{VI}^{Rep}$. Therefore, any $\tau \in (\tau_{VI},\tau_{VI}^{Rep})$ which would have the integration with $B$ now leads to the integration with $A$. Consequently, the entrant is more likely to suffer from lower profit if consumers are reluctant to disclose information to him. 

\begin{proposition}(\textbf{The Effect of Asymmetric Privacy Concern})
If a consumer faces higher privacy nuisance cost from disclosing information to the entrant who does not have market reputation, it may aggravate the entrant's disadvantage, thereby adversely affecting him in terms of lower market share and profit.
\end{proposition}

%Detailed proof is in the Appendix.

\section{Empirical Evidence}

Throughout the model, I implicitly assume that consumers are concerned about privacy protection when making a decision and their privacy concern is asymmetric with respect to firm's reputation. In order to see whether a consumer cares about privacy when choosing a product, in reality, and how likely consumer's willingness to provide information is trust-based, i.e., related to data collector's reputation, I analyze the mobile application ecosystem. Specifically, I used \textit{Python} to scrape necessary data from Google Play Store which is the official application(app) store for the Android operation system. Since users browse and download apps directly from this website, it provides all necessary app-specific information such as price, category, app size and so on. On top of that, the Google Play Store provides information on which permissions each app requests, so that users can see those permissions before downloading apps. As for the permissions, there are a number of different groups for each permission such as Device and App History, Identity, Contacts, Location and so on. For example, \textit{Google Photos} asks users to have access to three information about Identity, three about Contacts, two about Location, and 27 different information about other groups. Thus, users can see details about those permissions and decide whether to download apps or not.  

More to the point, considering that privacy concern is closely related to user's trust toward data collectors, such effect might be stronger for small app providers which do not have market reputation while app providers who have great reputation would not have any negative effects from unnecessary permissions. Therefore, the focus here is to examine whether there is any asymmetric effect of privacy concern with respect to firm's size or reputation even after controlling app-specific characteristics and some other relevant factors which affect app demand. 

\subsection{Data and Summary Statistics}

I have gathered the data from the end of October to the middle of December on a weekly basis. The data set contains top 540 ranked free apps for every weekend from the end of October and another top 540 ranked paid app for every weekend from the first of November, so the total sample size is 6,425.

First, an app demand measure and the number of privacy-related permissions each app requests are necessary variables to analyze the asymmetric effect of privacy concern on consumer's demand. As for the app demand, I use the number of reviews of each app as a proxy for the demand measure because it would be at least the lower bound of demand considering that some portion of customers who downloads each app writes a review.\footnote{Kummer and Schulte (2016) and Ghose and Han (2014) also partly use the number of reviews as a demand measure.} As for the number of permissions, Google provides total 17 categories of permission groups, such as Device \& App History, Identity, Contacts, Location and so on, which each app developer can choose from. Each app developer then might have several different permissions for each category. Among 17 categories, ``Other'' category is for manufacturer or app-specific custom settings which include relatively insignificant permissions to privacy. On the other hand,  the Figure~\ref{permission} shows that ``Identity (Identity and Contact)'', ``Location'', ``Social (SMS and Phone)'', and ``Device \& App History'' related permissions are the most frequent permission groups apps ask on average. %\footnote{I follow some of the permission group notations as in Kummer and Schulte (2016). } 
			\begin{figure}[!tbp]
	\centering
	\begin{minipage}[b]{0.46\textwidth}
		\includegraphics[width=\textwidth]{Free_Paid_Category}
		\caption{The most frequent permissions on average}\label{permission}
	\end{minipage}
	\hfill
	\begin{minipage}[b]{0.46\textwidth}
		\includegraphics[width=\textwidth]{Free_Paid_Com}
		%		\caption{Vertical Integration and Data Sharing}
		\caption{The number of privacy-sensitive permissions}\label{freepaidpermission}
	\end{minipage}
\end{figure}


Also, the data shows that there are some unnecessary permissions which do not affect apps' functioning. Those redundant permissions which might be used for monetizing purposes make consumers more reluctant to download apps due to privacy concerns.\footnote{For one thing, although GPS/Navigation app needs to access to location information to function properly, information on user's web browsing history would not be necessary.} Moreover, as in the Figure~\ref{freepaidpermission}, free apps ask more permissions than paid apps on average, which raises a doubt on that app developers have ulterior motives for providing free apps.


Last, I use a set of app-specific characteristics as control variables for estimating app demand. It includes price, app age since release dates, average app rating, app size, dummy for game apps, the number of screenshots on app description page, the total number of distinct apps each developer provides and dummy variable for top developer status. As for top developer variable, the variable is 1 if Google Play Store grants ``Top Developer'' status to an app.
%In the following empirical model, I focus on the number of redundant permissions and the top four group of permissions which are in the Figure 7 as the measure of privacy-sensitive permissions. 

The selected set of variables is summarized in the Table~\ref{sumstat} where $P\_$ denotes permissions-related variables. Also, $D$\_ denotes a dummy variable for whether an app asks at least one permission related to location, social, identity, and browsing history. I use log transformed variables for the number of reviews (as a demand measure) and for app age. 




\begin{table}[ht]\centering \caption{Summary Statistics \label{sumstat}}
	\begin{tabular}{l c c  c}\hline\hline
		\multicolumn{1}{c}{\textbf{Variable}} & \textbf{Mean}
		& \textbf{Std. Dev.} & \textbf{N}\\ \hline
		ln\_Num\_Reviews &  10.599 & 3.001  & 6425\\ \hline\hline
		ln\_App\_Age & 6.647 & 1.627  & 6422\\
		Price & 1.457 & 2.614  & 6425\\
		Game\_non\_game &  0.445 & 0.497  & 6425\\
		Avg\_rating &  4.298 & 0.363  & 6425\\
		Num\_of\_screen\_shots & 13.116 & 6.465  & 6425\\ \hline\hline
		%		Num\_of\_unique\_developer &  2.767 & 4.552  & 6425\\
		Num\_of\_apps\_for\_developer &  2.767 & 4.552  & 6425\\
		Dummy\_top\_developer & 0.468 & 0.499  & 6425\\ \hline\hline
		P\_In\_app\_purchase &  0.402 & 0.49  & 6425\\
		P\_Device\_and\_App\_history &  0.248 & 0.566  & 6425\\
		P\_Indentity &  0.555 & 0.786  & 6425\\
		P\_Contacts & 0.619 & 0.876  & 6425\\
		P\_Calendar & 0.063 & 0.323  & 6425\\
		P\_Location &  0.519 & 0.823  & 6425\\
%		P\_SMS &  0.223 & 0.842  & 6425\\
%		P\_Phone &0.613 & 1.002  & 6425\\
%		P\_PhotosMediaFiles & 1.706 & 0.843  & 6425\\
%		P\_Storage & 1.654 & 0.795  & 6425\\
%		P\_Camera & 0.245 & 0.435  & 6425\\
%		P\_Microphone &  0.16 & 0.366  & 6425\\
%		P\_Wifi\_Info & 0.624 & 0.484  & 6425\\
%		P\_Bluetooth\_Info & 0.001 & 0.033  & 6425\\
%		P\_Cellular\_data &  0.009 & 0.096  & 6425\\
%		P\_Wearable\_sensors & 0.001 & 0.033  & 6425\\
%		P\_Device\_ID\_Call & 0.416 & 0.495  & 6425\\
%		P\_Other &  6.639 & 5.474  & 6425\\
		D\_location  &0.316 & 0.465  & 6425\\
		D\_social & 0.578 & 0.494  & 6425\\
		D\_identity & 0.576 & 0.494  & 6425\\
		D\_history &  0.191 & 0.393  & 6425\\
		Total\_permissions & 14.698 & 10.244  & 6425\\
		Total\_without\_others &  8.059 & 5.429  & 6425\\
		Redundant\_permissions & 3.038 & 3.481  & 6424\\
		\hline\hline
	\end{tabular}
\end{table}

\subsection{Empirical Model and Result}


I estimate how the number of redundant permissions affects app demand based on a cross section sample. The preliminary empirical model is as follows.
\begin{align}
\text{ln\_reviews}_i = \alpha + \boldsymbol{\beta} D_{i}+\beta_5 \text{Redundant}_{i}+\gamma D_i^\text{Top\_Dev}\times\text{Redundant}_i +\delta Price_i + \boldsymbol{\theta}X_i +\epsilon_i
\end{align}
where $i$ denotes each app, $D_{i}$ denotes dummy variables for privacy related permissions (for location, social, identity, and browsing history), $Redundant_i$ means the total number of redundant permissions, $\gamma$ is the coefficient of interest which shows the interaction effect of top developer and the number of redundant permissions, $X_i$ is a set of app-specific characteristics as control variables.

The Table~\ref{Results1} shows the preliminary result based on a cross-section sample. The first column only includes free apps only while the second column includes paid apps only. The last column includes the full sample. 

\begin{footnotesize}
	\renewcommand{\arraystretch}{0.9}
	\begin{table}[htbp]\centering \caption{Preliminary Result from Cross Section Data \label{Results1}}
		\begin{tabular}{lccc} \hline
			& (1) & (2) & (3) \\
			VARIABLES & Free & Paid & Full Sample \\ \hline
			&  &  &  \\
			D\_Price &  &  & -4.358*** \\
			&  &  & (0.0574) \\
			ln\_App\_Age & 1.151*** & 0.363*** & 0.604*** \\
			& (0.0268) & (0.0193) & (0.0161) \\
			Num\_of\_apps\_for\_dev & 0.0530*** & -0.0345** & 0.0280*** \\
			& (0.00534) & (0.0159) & (0.00537) \\ 
			Game\_non\_game & 0.990*** & 0.516*** & 0.635*** \\
			& (0.0718) & (0.0850) & (0.0557) \\
			Avg\_rating & 1.874*** & 1.245*** & 1.487*** \\
			& (0.0937) & (0.0938) & (0.0680) \\
			Number\_of\_screenshots & 0.0307*** & 0.0329*** & 0.0434*** \\
			& (0.00491) & (0.00643) & (0.00402) \\
			D\_location & -0.184*** & 0.225* & 0.0168 \\
			& (0.0666) & (0.117) & (0.0602) \\
			D\_Social & -0.255 & 0.188 & -0.0145 \\
			& (0.248) & (0.267) & (0.188) \\
			D\_Identity & 0.535** & 0.665** & 0.742*** \\
			& (0.249) & (0.265) & (0.189) \\
			D\_History & 0.747*** & 0.309** & 0.577*** \\
			& (0.0747) & (0.150) & (0.0708) \\
			Redundant & -0.0223** & -0.0441* & -0.0650*** \\
			& (0.0111) & (0.0266) & (0.0108) \\
			Top\_Dev $\times$ Redundant & 0.0853*** & 0.276*** & 0.154*** \\
			& (0.0112) & (0.0276) & (0.0109) \\
			Constant & -4.370*** & -0.956** & 0.674** \\
			& (0.433) & (0.437) & (0.311) \\
			&  &  &  \\
			Observations & 3,639 & 2,130 & 5,769 \\
			R-squared & 0.482 & 0.333 & 0.630 \\ \hline
			\multicolumn{4}{c}{ Standard errors in parentheses} \\
			\multicolumn{4}{c}{ *** p$<$0.01, ** p$<$0.05, * p$<$0.1} \\
\end{tabular}\end{table}\end{footnotesize}

From the first column (1) for which only free apps are taken into consideration, I can see that location and social-contacts related permissions are associated with the lower number of reviews, which implies that those types of permissions have negative effects on app demand, though the estimate for social-related permissions is not statistically significant. However, the demand for paid apps is not adversely affected by those groups of privacy-related permissions. Also, as an app asks more redundant permissions which are not related to the app's functioning, it is associated with lower app demand for all types of apps. Interestingly, this effect is mitigated for apps launched by top developers who have greater market reputation. 

For example, from the full sample, one additional unit change of redundant permissions for non-top-developers who relatively do not have reputation corresponds to a decrease in app demand of 6.5\% approximately. However, as for top developers, it is rather associated with an increase in demand of 8.9\%. From the estimation results, it is shown that there is an asymmetric reputation effect of privacy concern with respect to firm's status in the market. This empirical evidence corroborates the assumptions which are imposed throughout the paper. 


\section{Policy Implication and Concluding Remarks}
In this paper, I analyze how consumer's privacy concern affects market competition where each seller attracts potential customers by making targeted ads based on personal information obtained from the platform. In particular, I focus on the relationship between privacy sensitivity and data-sharing aspects of vertical integration between the platform and seller. I show that the platform and incumbent with better initial targeting technology are more likely to vertically integrate as there are more privacy-sensitive consumers. The integrated firm always wants to foreclose the unaffiliated entrant from data access, thereby adversely affecting the entrant in terms of lower market share and profit. Therefore, the entrant who needs consumer data to overcome his initial disadvantage on targeting technology is disproportionately affected by lack of data arising from more privacy concern. On top of that, this eventually leads to lower consumer surplus as well as total social welfare due to the lack of competition arising from data foreclosure. Consequently, individually rational decision on information disclosure which depends on each consumer's privacy sensitivity might not be socially optimal when aggregated. This is likely to happen if consumers fail to take into consideration any unexpected effect of their privacy-related information disclosure decision on the relevant market competition. 

Therefore, any policy which encourages no vertical integration or integration with the entrant would be beneficial. In this sense, I propose some plausible policy remedy which leads to socially desirable outcomes. Given that $\tau$ is large enough (larger than $\tau_{VB}$) so that vertical integration can always emerge, any remedy that makes more people willing to disclose information or be more privacy-insensitive can lead to the integration with the entrant in equilibrium which is socially optimal. Privacy Certification Program can be one possible remedy for this purpose. If the government grants a specific seal indicating that firms comply with government-enacted privacy rules, marginal privacy-sensitive consumers who refuse to provide information due to any possible data abuse might switch to disclose personal information. Though there are a few privacy firms, such as \textit{TRUSTe}, which do the similar business, their certifications only indicate self-certified at the best. The government certification program can play a role as a global standard which helps participating firms gain more reputation in data usage. As shown in section 6, privacy is trust-based matter in a sense that consumers do care who asks for their personal information. Given that the willingness to disclose information depends on firm's reputation in reality, this remedy is likely to be in effect. 

There can be another policy implication concerning data foreclosure practice in vertical integration. Since consumer data has become a key to seller's business, data foreclosure is directly related to the competitive structure. This anti-competitive effect is more apparent in the integration with incumbent. To mitigate this detrimental effect, regulators might force the integrated firm to share their customer data with rivals by asking to set data price at a reasonable range.\footnote{For one thing, U.S. media companies plan to ask such data-sharing related regulation in response to AT\&T and Time Warner merger.} 

Overall, this paper expands the horizon of relevant literature by adding ``privacy'' and ``information'' to market competitive structure in which the entrant is competing with the incumbent. Still, enforcers have not established any concrete antitrust standards for big data and data-driven vertical integration though they are potentially important area to investigate. In that sense, the findings from my model help to understand the relevant issues and propose various policy implications regarding privacy protection and data-driven vertical integration. 


\begin{thebibliography}{99}
	\bibitem{pa} Acquisti, A., Varian, H.R., ``Conditioning Prices on Purchase History'', \textit{Marketing Science}, Vol. 24, No. 3, (2005) pp 367–381
	\bibitem{pa} Avi, G., Tucker, C.E., ``Privacy Regulation and Online Advertising'', \textit{Management Science}, Vol.57, No.1, (2011) pp 57-71
\bibitem{pa} Adjerid, I., Acquisti, A., Telang, R., Padman, R., Adler-Milstein, J., ``The Impact of Privacy Regulation and Technology Incentives: The Case of Health Information Exchanges'', \textit{Management Science}, Vol.62, No.4, (2016) pp 1042-1063 
\bibitem{pa} Belleflamme, P., Vergote, W. ``Monopoly Price Discrimination and Privacy: The Hidden Cost of Hiding'', \textit{Economics Letters}, Vol.149, (2016) pp 141-144

\bibitem{pa} Bergemann, D., Bonatti, A., ``Selling cookies'', \textit{American Economic Journal: Microeconomics}, Vol.7, No.3, (2015) pp 259-94
\bibitem{pa} Braulin, F.C., Valletti, T., ``Selling customer information to competing firms'', \textit{Economics Letters}, Vol.149 ,(2016) pp 10-14
\bibitem{pa} Campbell, J., Goldfarb, A., Tucker, C., ``Privacy regulation and market structure'', 
\textit{Journal of Economics and Management Strategy}, Vol.24, No.1, (2015) pp 47-73  
\bibitem{pa} Casadesus-Masanell, R., Hervas-Drane, A.,  ``Competing with privacy'', 
\textit{Management Science}, Vol.61, No.1, (2015) pp 229-246 
\bibitem{pa} Choi, J.P., Jeon, D.S., Kim, B.C.,  ``Privacy and Personal Data Collection with Information Externalities'', 
\textit{Working Paper}, (2016)
\bibitem{pa} Conitzer, V., Taylor, C.R., Wagman, L., ``Hide and Seek: Costly Consumer Privacy in a Market with Repeat Purchases'', \textit{Marketing Science}, (2012) pp 277-292
\bibitem{pa} Corni\`{e}re, A.d., Nijs, R.d., ``Online advertising and privacy'', 
\textit{RAND Journal of Economics}, Vol.47, No.1, (2016) pp 48-72
\bibitem{pa} Fudenberg, D., Tirole, J., ``Customer Poaching and Brand Switching'', \textit{RAND Journal of Economics}, Vol. 31(4), (2000) pp 634-657
\bibitem{pa} Fudenberg, D., Villas-Boas. J., ``Price Discrimination in the Digital Economy”, \textit{The Oxford Handbook of the Digital Economy}, (2012)
\bibitem{pa} Kim, J.H., Wagman, L., ``Screening incentives and privacy protection in financial markets: a theoretical and empirical analysis'', \textit{RAND Journal of Economics}, Vol.46, No.1, (2015) pp 1-22
\bibitem{pa} Kim, J.H., Wagman, L, Wickelgren, A. L., ``The Impact of Access to Consumer Data on the Competitive Effects of Horizontal Mergers'', \textit{Available at SSRN: https://ssrn.com/abstract=2728378 or http://dx.doi.org/10.2139/ssrn.2728378}, (2016)
\bibitem{pa} Koh, B., Raghunathan, S., Nault, B.R., ``Is voluntary profiling welfare enhancing?'', MIS Quarterly, Vol. 41, Issue 1, (2017)
\bibitem{pa} Kox, H., Straathof, B., Zwart, G., ``Targeted advertising, platform competition and privacy'', CPB Discussion Paper 280, (2014)
\bibitem{pa} Kummer, M.E., Schulte, P., ``When Private Information Settles the Bill: Money and Privacy in Google's Market for Smartphone Applications'', \textit{ZEW - Centre for European Economic Research Discussion Paper}, No.16-031, (2016) 
\bibitem{} Levin, J., Milgrom, P.,  ``Online Advertising: Heterogeneity and Conflation Market Design'', \textit{ American Economic Review: Papers \& Proceedings}, 100, (2010) pp 603-607
\bibitem{pa} Posner, R.,``The Economics of Privacy'', \textit{American Economic Review}, Vol.71(2), (1981) pp 405-409.
\bibitem{pa} Shy, O., Stenbacka, R., ``Customer Privacy and Competition'', 
\textit{Journal of Economics and Management Strategy}, Vol.25, Issue.3, (2016) pp 539-62
\bibitem{pa} Stigler, G.,``An Introduction to Privacy in Economics and Politics'', \textit{The Journal
of Legal Studies}, Vol 9(4), (1980) pp 623-644
\bibitem{pa} Taylor, C., ``Consumer Privacy and the Market for Customer Information'', \textit{RAND
Journal of Economics}, Vol. 35(4), (2004) pp 631-650.

\bibitem{pa} Taylor, C. and L. Wagman, ``Customer Privacy in Oligopolistic Markets: Winners,
Losers, and Welfare'', \textit{International Journal of Industrial Organization}, Vol. 34, (2014) pp 80-84
\bibitem{pa} Tucker, C.E., ``Social Networks, Personalized Advertising, and Privacy Controls'', 
\textit{Journal of Marketing Research}, Vol.51, No.5, (2014) pp 546-562      
\bibitem{pa} Villas-Boas, J., ``Dynamic Competition with Customer Recognition”, \textit{RAND Journal
of Economics}, Vol. 30(4), (1999) pp 604-631  
\bibitem{pa} Villas-Boas, J., ``Price Cycles in Markets with Customer Recognition”, \textit{RAND Journal
	of Economics}, Vol. 35(3), (2004) pp 486-501
\end{thebibliography}


	\end{document}